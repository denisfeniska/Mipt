\documentclass{article}
\usepackage[utf8]{inputenc}
\usepackage{graphicx} % Required for inserting images
\usepackage[english,russian]{babel}
\usepackage{hyperref}
\usepackage{amssymb}

\begin{document}

\section*{Метрические методы. Метрики качества \newline 10 баллов. +2 бонусных балла}

\subsection*{Задача 1. (1 балл)}

    Оценим время работы алгоритма ближайших соседей по количеству операций. Пусть $X$~--- обучающая выборка размера $n$, $Y$~--- тестовая выборка размера $m$. Размерность признакового пространства $d$. 
    
    \noindent
    Таким образом, $X \in \mathbb{R}^{n \times d}$, а $Y \in \mathbb{R}^{m \times d}$.
    
    \noindent
    Квадрат евклидова расстояния между объектами $x_i$ и $y_j$ записывается как:
    $$
        \rho(x_i, y_j) = \sum_{k=1}^d (x_i^k - y_j^k)^2.        
    $$
    \begin{itemize}
        \item Определите количество операций, необходимое для подсчета всех попарных расстояний в наивном случае.
        \item Предложите способ, с помощью которого можно уменьшить количество операций. Оцените количество операций для предложенного метода.
    \end{itemize}

\subsection*{Задача 2. (2 балла)}

    Дано $n$ объектов, распределённых равномерно внутри $d$-мерного единичного шара с центром в нуле.
    \begin{itemize}
        \item Найдите выражение для медианы расстояния от начала координат до ближайшего объекта.
        \item Проинтерпретируйте полученный результат в терминах применимости метода ближайшего соседа в различных ситуациях.
    \end{itemize}
    
    \noindent
    Считайте, что метрика в задаче евклидова.
    
    \noindent
    \emph{Указание: попробуйте смоделировать событие и посчитать его вероятность в терминах функций распределения.}

\subsection*{Задача 3. (3 балла)}

    Решается задача классификации с помощью алгоритма ближайших соседей (метрика евклидова). Для тестового объекта $z$ ближайшим соседом с расстоянием $\rho_x$ является $x$, вторым ближайшим соседом с расстоянием $\rho_y$ является объект $y$. Остальные объекты обучающей выборки находятся от $z$ на достаточно большом расстоянии.

    Ко всем объектам добавляется новый признак: для $z$ и $y$ значение признака распределёно равномерно на отрезке $[-1, 1]$, для всех остальных объектов значение признака равно нулю. 
    \begin{itemize}
        \item Посчитайте вероятность того, что теперь ближайшим соседом для $z$ будет не $x$, а $y$.
        \item Проинтерпретируйте полученный результат в терминах применимости метода ближайших соседей.
    \end{itemize}

    \noindent
    \emph{Указание: возможно в этой задаче пригодится знание криволинейных интегралов.}

\subsection*{Задача 4. (1 балл)}

    Докажите, что ROC-AUC случайного классификатора равен $0.5$.

\subsection*{Задача 5. (2 балла)}

    Пусть, $a = a(x)$ ответ алгоритма. На сколько может уменьшиться ROC-AUC при использовании функции $\min(a, 0.5)$ над оценками алгоритма?

\subsection*{Задача 6. (3 балла)}

    Подробнее ознакомьтесь с материалом по ROC-AUC по \href{https://alexanderdyakonov.wordpress.com/2017/07/28/auc-roc-%D0%BF%D0%BB%D0%BE%D1%89%D0%B0%D0%B4%D1%8C-%D0%BF%D0%BE%D0%B4-%D0%BA%D1%80%D0%B8%D0%B2%D0%BE%D0%B9-%D0%BE%D1%88%D0%B8%D0%B1%D0%BE%D0%BA/}{ссылке} и решите следующую задачу:

    \noindent
    Пусть на ответах алгоритма $m$ (принимающих значения от $0$ до $1$) задано распределение объектов класса $1$ (доля объектов класса $1$ в зависимости от ответа алгоритма) следующим образом:
    $$
        \mathbb{P} (m \in [a, b] \ | \ y = 1) = \int_a^b p(z)dz.
    $$

    \noindent
    Распределение объектов класса 0 задаётся так:
    $$
        \mathbb{P} (m \in [a, b] \ | \ y = 0) = \int_a^b (2 - p(z))dz,
    $$
    где $p(z) = -1.5z^2 + 3z$.

    \noindent
    Найдите вероятностные оценки на величины TPR, FPR и ROC-AUC.

\end{document}
